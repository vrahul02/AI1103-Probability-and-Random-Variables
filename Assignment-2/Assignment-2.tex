\documentclass[journal,12pt,twocolumn]{IEEEtran}

\usepackage{setspace}
\usepackage{gensymb}
\singlespacing
\usepackage[cmex10]{amsmath}

\usepackage{amsthm}

\usepackage{mathrsfs}
\usepackage{txfonts}
\usepackage{stfloats}
\usepackage{bm}
\usepackage{cite}
\usepackage{cases}
\usepackage{subfig}

\usepackage{longtable}
\usepackage{multirow}

\usepackage{enumitem}
\usepackage{mathtools}
\usepackage{steinmetz}
\usepackage{tikz}
\usepackage{circuitikz}
\usepackage{verbatim}
\usepackage{tfrupee}
\usepackage[breaklinks=true]{hyperref}
\usepackage{graphicx}
\usepackage{tkz-euclide}

\usetikzlibrary{calc,math}
\usepackage{listings}
    \usepackage{color}                                            %%
    \usepackage{array}                                            %%
    \usepackage{longtable}                                        %%
    \usepackage{calc}                                             %%
    \usepackage{multirow}                                         %%
    \usepackage{hhline}                                           %%
    \usepackage{ifthen}                                           %%
    \usepackage{lscape}     
\usepackage{multicol}
\usepackage{chngcntr}

\DeclareMathOperator*{\Res}{Res}

\renewcommand\thesection{\arabic{section}}
\renewcommand\thesubsection{\thesection.\arabic{subsection}}
\renewcommand\thesubsubsection{\thesubsection.\arabic{subsubsection}}

\renewcommand\thesectiondis{\arabic{section}}
\renewcommand\thesubsectiondis{\thesectiondis.\arabic{subsection}}
\renewcommand\thesubsubsectiondis{\thesubsectiondis.\arabic{subsubsection}}


\hyphenation{op-tical net-works semi-conduc-tor}
\def\inputGnumericTable{}                                 %%

\lstset{
%language=C,
frame=single, 
breaklines=true,
columns=fullflexible
}
\begin{document}

\newcommand{\BEQA}{\begin{eqnarray}}
\newcommand{\EEQA}{\end{eqnarray}}
\newcommand{\define}{\stackrel{\triangle}{=}}
\bibliographystyle{IEEEtran}
\raggedbottom
\setlength{\parindent}{0pt}
\providecommand{\mbf}{\mathbf}
\providecommand{\pr}[1]{\ensuremath{\Pr\left(#1\right)}}
\providecommand{\qfunc}[1]{\ensuremath{Q\left(#1\right)}}
\providecommand{\sbrak}[1]{\ensuremath{{}\left[#1\right]}}
\providecommand{\lsbrak}[1]{\ensuremath{{}\left[#1\right.}}
\providecommand{\rsbrak}[1]{\ensuremath{{}\left.#1\right]}}
\providecommand{\brak}[1]{\ensuremath{\left(#1\right)}}
\providecommand{\lbrak}[1]{\ensuremath{\left(#1\right.}}
\providecommand{\rbrak}[1]{\ensuremath{\left.#1\right)}}
\providecommand{\cbrak}[1]{\ensuremath{\left\{#1\right\}}}
\providecommand{\lcbrak}[1]{\ensuremath{\left\{#1\right.}}
\providecommand{\rcbrak}[1]{\ensuremath{\left.#1\right\}}}
\theoremstyle{remark}
\newtheorem{rem}{Remark}
\newcommand{\sgn}{\mathop{\mathrm{sgn}}}
\providecommand{\abs}[1]{\vert#1\vert}
\providecommand{\res}[1]{\Res\displaylimits_{#1}} 
\providecommand{\norm}[1]{\lVert#1\rVert}
%\providecommand{\norm}[1]{\lVert#1\rVert}
\providecommand{\mtx}[1]{\mathbf{#1}}
\providecommand{\mean}[1]{E[#1]}
\providecommand{\fourier}{\overset{\mathcal{F}}{ \rightleftharpoons}}
%\providecommand{\hilbert}{\overset{\mathcal{H}}{ \rightleftharpoons}}
\providecommand{\system}{\overset{\mathcal{H}}{ \longleftrightarrow}}
	%\newcommand{\solution}[2]{\textbf{Solution:}{#1}}
\newcommand{\solution}{\noindent \textbf{Solution: }}
\newcommand{\cosec}{\,\text{cosec}\,}
\providecommand{\dec}[2]{\ensuremath{\overset{#1}{\underset{#2}{\gtrless}}}}
\newcommand{\myvec}[1]{\ensuremath{\begin{pmatrix}#1\end{pmatrix}}}
\newcommand{\mydet}[1]{\ensuremath{\begin{vmatrix}#1\end{vmatrix}}}
\numberwithin{equation}{subsection}
\makeatletter
\@addtoreset{figure}{problem}
\makeatother
\let\StandardTheFigure\thefigure
\let\vec\mathbf
\renewcommand{\thefigure}{\theproblem}
\def\putbox#1#2#3{\makebox[0in][l]{\makebox[#1][l]{}\raisebox{\baselineskip}[0in][0in]{\raisebox{#2}[0in][0in]{#3}}}}
     \def\rightbox#1{\makebox[0in][r]{#1}}
     \def\centbox#1{\makebox[0in]{#1}}
     \def\topbox#1{\raisebox{-\baselineskip}[0in][0in]{#1}}
     \def\midbox#1{\raisebox{-0.5\baselineskip}[0in][0in]{#1}}
\vspace{3cm}
\title{AI1103 Assignment-2}
\author{V Rahul - AI20BTECH11030}
\maketitle
\newpage
\bigskip
\renewcommand{\thefigure}{\theenumi}
\renewcommand{\thetable}{\theenumi}
Download all python codes from 
\begin{lstlisting}
https://github.com/vrahul02/AI1103-Probability-and-Random-Variables/tree/main/Assignment-2/Codes
\end{lstlisting}
%
and latex-tikz codes from 
%
\begin{lstlisting}
https://github.com/vrahul02/AI1103-Probability-and-Random-Variables/tree/main/Assignment-2/Assignment-2.tex
\end{lstlisting}
\section*{Problem Gate-28}
Consider two independent random variables X and Y with identical distributions. The variables X and Y take value 0,1 and 2 with probabilities $\frac{1}{2}$, $\frac{1}{4}$ and $\frac{1}{4}$ respectively. What is the conditional probability $\Pr\brak{X+Y=2\: |\: X-Y=0}$?\\\\
a) 0\\
b) $\frac{1}{16}$\\
c) $\frac{1}{6}$\\
d) 1
\section*{Solution}
The values that the random variable X can take along with its probabilities are given by
\begin{table}[h]
\centering
\begin{tabular}{|l|l|l|l|}
\hline
X             & 0             & 1             & 2             \\ \hline
$\Pr\brak{X}$ & $\frac{1}{2}$ & $\frac{1}{4}$ & $\frac{1}{4}$ \\ \hline
\end{tabular}
\end{table}
\\
The values that the random variable Y can take along with its probabilities are given by
\begin{table}[h]
\centering
\begin{tabular}{|l|l|l|l|}
\hline
Y             & 0             & 1             & 2             \\ \hline
$\Pr\brak{Y}$ & $\frac{1}{2}$ & $\frac{1}{4}$ & $\frac{1}{4}$ \\ \hline
\end{tabular}
\end{table}
\\
X and Y are independent random variables\\
Possible outcomes of the event are
\begin{align}
S=\{(0,0),(0,1),(0,2),\notag\\(1,0),(1,1),(1,2),\notag\\(2,1),(2,2),(2,3)\}
\end{align}
The outcomes where X+Y=2 is given by set
\begin{align}
A=\{(0,2),(1,1),(2,0)\}
\end{align}
The outcomes where X-Y=0 is given by set
\begin{align}
B=\{(0,0),(1,1),(2,2)\}\\
A+B=\{(1,1)\}\\\notag\\
\Pr\brak{A+B}=\frac{1}{4}\times\frac{1}{4}=\frac{1}{16}\\\notag\\
\Pr\brak{B}=\frac{1}{2}\times\frac{1}{2}+\frac{1}{4}\times\frac{1}{4}+\frac{1}{4}\times\frac{1}{4}\notag\\
=\frac{1}{4}+\frac{1}{16}+\frac{1}{16}=\frac{6}{16}\\\notag\\
\Pr\brak{A\:|\:B}=\frac{\Pr\brak{A+B}}{\Pr\brak{B}}=\frac{\frac{1}{16}}{\frac{6}{16}}=\frac{1}{6}
\end{align}
\\Thus option c) is correct\\
\end{document}